\documentclass[a4paper]{article}
\usepackage[utf8]{inputenc}
\usepackage[T1]{fontenc}
\usepackage{url}
\usepackage[]{algorithm2e}

\title{2-29-2 Graph Mining - Event Detection}
\author{Yann Ramusat}

\begin{document}
\maketitle
The whole project is done in C++. It is freely available on Github\footnote{https://github.com/yannramusat/Extractor}. Place your own datasets in a directory \textit{data} where you have installed the program. Use -h to get helped if needed. 
You need to modify the $15^{th}$ line of \textit{main.cpp} and recompile to change the dataset you want to use another one.

\tableofcontents

\section{Choices made}
\subsection{Part 4}

 To identify locations with a peak we use the following function:
\[
   \rho*coeff + coeff^2
\]
 where $\rho$ is the average number of peak per slot for the location and $coeff$ is an integer choosed by the user (default is $9$). 
 
 The first part is basically used to identify a peak and the quadratic part is to avoid locations with not enough tweets (for example if the average number of tweets per day for a location is less than one, $5$ tweets for a day could be considered as a peak).
 
 You can change the number of slot per day in order to be more precise in finding a peak by using the $-d$ option. Per default, the program uses day as frequence ($d=1$). 
 
 Experimentally (with $d=1$), it seems that $8-10$ is a really suitable range of values for $coeff$ to avoid finding other events than crisis. For value $9$ over the dataset \textit{marchCrisis} the algorithm identify locations (day): belgium (22), brussels (22), capitol (28), chicago (12), cyprus (29), istanbul (19), lahore (27 - 28), metro (22), nigeria (25), pakistan (27) and zaventem (22).
 
 You can easily verify that all of this peaks corresponds to real events that have been widely discussed in the press.
 
 For smaller values (like $5$) we detect much more events that we might not have heard from, like spain (25) with keywords "injured, killed, crash, students, bus" precisely explaining what happened. But we also have some false positives like the concert of Justin Bieber or the discourse of Barack Obama at Havana (Cuba).
 
 Choosing values of $coeff$ and the number of slots per day is a tradeoff between avoiding false positives and finding events not covered by mainstream medias.
 
\subsection{Part 5}

\begin{algorithm}
\KwData{A weighted graph $G$}
 \KwResult{A subgraph $H$ with high density}
	$H = G$\;
\While{$G$ contains at least one edge}{
  Let $\nu$ be the node with minimum degree $\delta_G (\nu)$ in $G$\;
  Remove $\nu$ and all its edges from $G$\;
  If $\rho(H) > \rho(G)$ then $H \leftarrow G$\;
  }
  return $H$\;
 \caption{Densest Subgraph}
\end{algorithm}

For each location with a peak we construct the co-occurence graph and use a generalization of the Densest Subgraph Algorithm we have seen in Lesson $2$ because we consider a weighted graph (with non-negative integer weights).

We then apply a simple heuristic keeping only the five keywords of maximum degree and return them.

Note that we filter the words whilst creating the co-occurence graph. The words belonging to the semantic vocabulary of meditation like "pray" or "thoughts" are ignored. Sometimes, this leads to less than $5$ words returned.

The complexity is linear but for some locations with lots of related tweets, for example Brussels in the dataset \textit{marchCrisis} there are $\sim 20K$ related tweets and a total edge weight of $\sim 388K$, so the mining can take up to $1$ minute to complete.

\section{Analysis}
\subsection{Precision}
\paragraph{Evaluation}
To evaluate the precision of the approach we'll refer to the following wikipedia\footnote{\url{https://fr.wikipedia.org/wiki/Liste_d'attentats_meurtriers}} page and use the dataset \textit{marchCrisis}. 
With parameters $e=5$ and $d=1$ we found $9$ out of $10$ terrorist attacks repertoried for this month. Remark that we didn't found the attack in Iraq (29). This might be explained by the fact that each location is uniquely computed and it was the second attack for the month in the same country.

We highlight the keywords found for the attack in Iraq (26): "bomber, football, kids, died, attack". They give a precise insight of the event (cf wikipedia): "L'attentat du 25 mars 2016 à Al-Asriya en Irak désigne un attentat-suicide à la bombe s'étant déroulé lors d'un match de football de jeunes...".

\paragraph{Events found}

We have already mentionned some events (and their keywords) we have found for march.

We now use the dataset for may.
Those are interresting: guam 18 american base bomber crashes b-52 and egypt 19 technical egyptair attack terror crash.
Both are about airplane crashes but we see some differencies over keywords giving specific information (crash in a base and attack or technical error for the other one).

\paragraph{Redundancies}

Because many locations can be associated with an event it's really difficult to avoid redundancies.
For example with the terrorist attacks of Brussels we have $4$ locations, all are corrects: the country (Belgium), the city (Brussels), the name (Zaventem) and the kind of public transports (Metro).

Redundancies in the list of keywords for a location generally do not occurs, the highlighted words are usually complementary.

\subsection{Interesting detections}

\subsubsection{Events not covered by mainstream medias}

For may our program found this: Munich 10 killed knife injured man station.
We found that there were a knife attack on may $10^{th}$ in a train station of Munich\footnote{https://en.wikipedia.org/wiki/Munich knife attack}. This can be easily found on the web using the keywords our algorithm returned but relatively harder using only the location or the date.

\subsubsection{Unknown facts about well-known events}

Unknown facts are difficult to find because we have only $5$ keywords to describe the event. But sometimes we have some precisions that are not widely discussed in the web (for example the fact that the crash in Guam was with a B-$52$).

It's usefull to use the keywords with the location when googling it in order to find the most precise source of information about the event.

\end{document}
